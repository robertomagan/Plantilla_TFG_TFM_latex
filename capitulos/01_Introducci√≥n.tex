\chapter{Introducción}\label{cap:introduccion}
%\minitoc

[La introducción tiene que poner en contexto al lector contando, a modo de historia, el origen y contexto del problema, motivando por qué es necesario abordarlo y finalizando con lo que se propone en el proyecto.]
\section{Motivación}
[Opcional si se ha motivado la realización del proyecto en los párrafos anteriores.]

\section{Objetivos}\label{sec:objetivos}

[Definir el objetivo principal del proyecto así como objetivos secundarios]

\begin{itemize}
    \item \textbf{OB1}. Bla bla bla ...
    \item \textbf{OB2}. Bla bla bla ...
    \item \textbf{OB3}. Bla bla bla ...
\end{itemize}

\section{Organización}

Este trabajo esta organizado en X capítulos. En el primer y actual Capítulo \ref{cap:introduccion}, se presenta el trabajo aportando la motivación y los objetivos deseados.

Bla bla bla ...

En el Capítulo~\ref{cap:conclusiones} ofrecemos las conclusiones del presente trabajo junto a las contribuciones y posibles retos futuros.