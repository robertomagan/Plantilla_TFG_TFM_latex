\chapter{Estado del arte}\label{cap:estado_del_arte}
[En el estado del arte se necesita hacer un estudio tanto sobre la tecnología que soporta el proyecto como sobre el problema que se aborda en él. Se puede estructurar por secciones y se aconseja utilizar referencias a los documentos e información que se describe aquí. 

Como norma general y más en proyectos con carácter investigador, se recomienda añadir un párrafo por cada documento/referencia que estudie del estado del arte, finalizando esta sección con un párrafo explicativo de la novedad/característica que propone, modifica o añade el proyecto sobre dicho estado del arte.]

\section{Sección}\label{sec:seccion}
Ejemplo de cita bibliográfica~\cite{leo_federated_2014}. Para añadir nuevas citas utilizar algún gestor de referencias bibliográficos como Zotero, exportar la referencia en formato BibTex y añadirla al fichero \texttt{referencias.bib}



\subsection{Sub-seccion}\label{sec:subsection}

